\documentclass[twocolumn]{article}
\usepackage{titling}
\usepackage{color}
\usepackage{tikz}
\usepackage{drawstack}
\usepackage{todonotes}
\usepackage{amsthm}

\usetikzlibrary{graphs, automata, quotes}

\newtheorem{principle}{Principle}
\newtheorem{corollary}{Corollary}
\newtheorem{claim}{Claim}

\title{One-pager 2}
\author{Martin \textsc{Vassor}}

\predate{}
\postdate{}
\date{}

\setlength{\droptitle}{-90pt}

\begin{document}
\maketitle
\section{TCP/IP and the end-to-end argument}
In order to argue on the potential violation of the \emph{end-to-end principle}, one need to precisely define it first.  
\begin{principle}[End-to-end]
	Given a property $P$ defined with respect to a set of properties $\mathcal{P}$, to correctly implement $P$ at layer $L_n$, the layer $L_{n-1}$ must provide $\mathcal{P}$.
\end{principle}
\begin{corollary}
Any try to provide $P$ at a lower level than $L_n$ is incorrect and unnecessary.
\end{corollary}
In my opinion, providing $P$ (or a partial/weaker version) in a lower layer does not violate the \emph{end-to-end principle}, as the layering principle defines only a minimal requirement on the abstractions, to allow substitution of layers. However, one can define such a point of view in a \emph{strict end-to-end principle}:
\begin{principle}[Strict end-to-end]
	End-to-end principle + each layer implements only the requirements of the above layer.
\end{principle}

\textsc{Tcp} is designed to provide an ordered, reliable link over an unreliable link, which is not the case of the specification of \textsc{ip}. Hence, providing partial reliability (even for efficiency) in the \emph{network layer} is redundant, and violate the \emph{strict end-to-end principle}, but not the \emph{general end-to-end principle}.

\section{NAT and the end-to-end argument}
\textsc{Nat}s are used to use a single \textsc{ip} address for many endpoints. Naively, it can be defined as a function $NAT: IP\times ports \mapsto IP\times ports$, implemented in the \emph{layer}.

But the concept of \emph{address} is only an inner view of the \emph{network layer}. From a point of view of an \emph{endpoint}, both configurations are \emph{indistinguishable}. Hence, in my opinion, \textsc{nat} violate \emph{layering separation}, but not \emph{end-to-end argument}\footnote{The argument involved is the same as the one to explain why in-network caches violate \emph{layering separation} but not \emph{end-to-end principle}.}. More generally: 
\begin{claim}
	Given a configuration $C_1$ which respects the \emph{end-to-end principle}, if a configuration $C_2$ is \emph{indistinguishable} from an endpoint point of view, then $C_2$ respects the \emph{end-to-end principle}.
\end{claim}

\section{Centralization versus Decentralization in control planes}
One can claim that distribution is more complex, requires more guarantees from the nodes\footnote{Typically requires a distributed agreement, which is a strong hypothesis.} but leads to a more scalable and more reliable solution than a centralized implementation\footnote{A centralized protocol is more prone to single point of failure, etc.}. Nevertheless, those are points which are valid for the \emph{centralized versus distributed} debate in general, and not specific for the control plane.

About the control plane, centralization allows a complete control on the routing algorithm. Also, as soon as the server is correct, verifying the correct behaviour of each individual router and endpoint on the network is easier.

But, in my opinion, decentralization provides better guarantees. In large scale systems such as the Internet, one should aim at reliability, and in particular partitioning.

Finally, a non technical argument in favour of distribution is the elegance of the algorithm: relying on the emergence of a global behaviour while each node of the network mesh only do local computations and communications.

\vfill
{\color{gray} \noindent Word count: 187 + 129 + 177 = 493, including footnotes, titles, caption, excluding references}

\end{document}


% \color{gray} (word count: 500, including footnotes, titles, caption, excluding references)
