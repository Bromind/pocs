\documentclass[twocolumn]{article}
\usepackage{titling}
\usepackage{color}
\usepackage{tikz}
\usetikzlibrary{shapes}
\usepackage{subcaption}

\definecolor{gray}{gray}{0.5}

\title{One-pager 5}
\author{Martin \textsc{Vassor}}

\predate{}
\postdate{}
\date{}

\setlength{\droptitle}{-90pt}

\newcommand{\vm}[0]{\textsc{vm}}
\newcommand{\vmm}[0]{\textsc{vmm}}
\newcommand{\os}[0]{\textsc{os}}

\begin{document}
\maketitle

\section{Introduction}
The question of comparing \emph{processes} and \emph{virtual machines} can be addressed in two --equivalent-- ways: from the viewpoint of a process (resp. \vmm), what is expected from the \os{} (resp. \vmm)? Or from the viewpoint of the \os{} (resp. \vmm), what should be provided? The approach of this proposal is the latter.

\paragraph{Models}
As \vmm{}s and \os{}es can be implemented in many ways. To provide a general answer, I consider minimal definitions for both systems: micro-kernel (as in \cite{liedtke_micro-kernel_1995}) and \vmm{} as in \cite{popek_formal_1974}.
\textsc{Liedtke} claims two requirements for micro-kernels: \emph{independence} and \emph{integrity}. This proposal shows that \vmm{}s ensure the same properties. This equivalence results in equivalent exposed interfaces, hence similar \emph{process} and \emph{virtual machine} foundations.

\section{Independence}
\label{sec:indep}
Each arbitrary \emph{subsystem} have to be implementable such that it is not corrupted nor disturbed by another subsystem. In both systems, this is guaranteed by virtual memory and pre-emptive scheduling. 

\paragraph{Virtual memory} In a micro-kernel, virtual memory is aided by hardware (\textsc{tlb} and page-walk) and in the \vmm{} abstraction, this is guaranteed by the usage of \emph{relocation-bounds register}.


\paragraph{Pre-emptive scheduling} Both systems have the possibility to preempt and abort subsystem.  An example to achieve that is to share the computing time in timeslices managed by hardware counters.

\section{Integrity} 
\label{sec:int}
To cooperate, processes should be able to exchange messages safely. Micro-kernels have to implement a trusted \textsc{ipc}. Integrity is defined as the following: for any two processes $P_1$ and $P_2$, if $P_1$ knows $P_2$, $P_1$ should be able to create a channel to $P_2$ which is neither corrupted nor eavesdropped. 

In \vmm{}s, \vm{}s are isolated and the only knowledge they have of other \vm{}s is via the network card (if any). All communication is done via the network, and the \vm{} does not distinguish an other \vm{} from a real distance machine. Integrity is hence ensured depending on the network properties.

\section{Limits \& conclusion}
\paragraph{Limits:}
While Sections~\ref{sec:indep} and~\ref{sec:int} provide strong similitudes between \os{}es and \vmm{}s, there are still some differences: 
\begin{description}
	\item[Exposed interface] \os{}es provide processes an ideal machine abstraction while \vmm{}s provide a real machine abstraction. 
	\item[Other kernels] Regular \os{}es provide much more features than micro-kernels such as filesystems, etc. Which have not been studied here. However, \vmm{}s have to provide at least basic equivalents (for instance virtual hard drive), etc.
\end{description}

\paragraph{Conclusion:}
\textsc{Os}es and \vmm{}s fundamentally serve the same purpose: providing a machine abstraction to upper layers (see Figure~\ref{fig:os} and~\ref{fig:vmm_as_os}). The difference lies in the usage of the abstraction: \os{}es provide an ideal abstract machine (for programming purposes) and \vmm{}s provide an exact hardware machine abstraction. 


\begin{figure}[b]
	\centering
	\begin{subfigure}{0.25\textwidth}
		\centering
		\begin{tikzpicture}
			\path [fill=red!20] (0,0) rectangle (4,1) node[pos=.5] {Physical machine};
			\path [fill=green!20] (0,1) rectangle (2,2) node[pos=.5] {\os};
			\path [fill=blue!20] (0,3) rectangle (4,2) node[pos=.5] {Process} rectangle (2,1);
			\draw [red, very thick] (0,2) node[left, text width = .6\textwidth] {Ideal machine interface} -- (2,2);
		\end{tikzpicture}
		\caption{\os}
		\label{fig:os}
	\end{subfigure}
	
	\begin{subfigure}{0.25\textwidth}
		\centering
		\begin{tikzpicture}
			\path [fill=red!20] (0,0) rectangle (4,1) node[pos=.5] {Physical machine};
			\path [fill=green!20] (0,1) rectangle (2,2) node[pos=.5] {\vmm};
			\path [fill=yellow!20] (0,3) rectangle (3,2) node[pos=.5] {\os} rectangle (2,1);
			\path [fill=blue!20] (0,4) rectangle (4,3) node[pos=.5] {Process} rectangle (3,1);
			\draw [red, very thick] (0,2) node[left, text width = .6\textwidth] {Real machine interface} -- (2,2);
			\draw [red, very thick] (0,3) node[left, text width = .6\textwidth] {Ideal machine interface} -- (3,3);
		\end{tikzpicture}
		\caption{\vmm}
		\label{fig:vmm}
	\end{subfigure}

	\begin{subfigure}{0.25\textwidth}
		\centering
		\begin{tikzpicture}
			\path [fill=red!20] (0,0) rectangle (4,1) node[pos=.5] {Physical machine};
			\path [fill=green!20] (0,1) rectangle (2,2) node[pos=.5] {\vmm};
			\path [fill=blue!20] (0,3) rectangle (4,2) node[pos=.5] {\os{} $+$ Processes} rectangle (2,1);
			\draw [red, very thick] (0,2) node[left, text width = .6\textwidth] {Real machine interface} -- (2,2);
		\end{tikzpicture}
		\caption{\vmm{} as an \os}
		\label{fig:vmm_as_os}
	\end{subfigure}
	\caption{From the point of view of the \vmm, the \emph{process} and \os{} layers are indistinguishable (\ref{fig:vmm} is view as \ref{fig:vmm_as_os}), resulting in the same pattern than a regular \os.}
	\label{fig}
\end{figure}


{\color{gray} (word count: 499 including section titles, caption, excluding references)}

\bibliographystyle{plain}
\bibliography{refs}
\end{document}
